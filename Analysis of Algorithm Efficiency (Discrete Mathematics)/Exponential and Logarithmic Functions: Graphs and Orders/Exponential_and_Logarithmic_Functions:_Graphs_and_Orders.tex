{\bf Exponential and Logarithmic Functions: Graphs and Orders}
\vskip 1mm
\hrule

\vskip 3mm
{\bf Graphs of Exponential Functions}
The exponential function with base $b>0$ is the function that send each real number $x$ to $b^x$. The graph of any exponential function with base $b>1$ has a shape that is similar to the graph of the exponential function with base $2$. If $0<b<1$, then ${1\over b}>0$ and the graph of the exponential function with base $b$ is the reflection across the vertical axis of the exponential function with base ${1\over b}$.

\filbreak
\vskip 1cm
{\bf Graphs of Logarithmic Functions}

\vskip 3mm
{\bf Definition}
\vskip 1mm
If $b$ is a positive real number not equal to $1$, then the {\bf logarithmic function with base $b$, $\log_b:R^+\to R$}, is the function that sends each positive real number $x$ to the number $\log_bx$, which is the exponent to which $b$ must be raised to obtain $x$. 

\vskip 1mm
The logarithmic function with base $b$ is the inverse of the exponential function with base $b$. It follows that the graphs of the two functions are symmetric with respect to the line $y=x$.

\vskip 2mm
If its base $b$ is greater than $1$, the logarithmic function is increasing. Analytically, this means that

\vskip 2mm
If $b>1$, then all positive numbers $x_1$ and $x_2$,

$$\hbox{if $x_1<x_2$, then $\log_b(x_1)<\log_b(x_2)$}$$

\filbreak
\vskip 1cm
{\bf Example 1: Base $2$ Logarithms of Numbers between Two Consecutive Powers of $2$}

\vskip 3mm
Provde the following Property:

\vskip 1mm
If $k$ is an integer and $x$ is a real number with

$$2^k\leq x < 2^{k+1}, \hbox{then }\lfloor\log_2x\rfloor=k$$

\vskip 3mm
{\bf Solution}
\vskip 1mm
{\bf Proof:}
\vskip 1mm
Suppose that $k$ is an integer and $x$ is a real number with

$$2^k\leq x<2^{k+1}$$

Because the logarithmic function with base $2$ is increasing, this implies that

$$\log_2(2^k)\leq\log_2x<\log_2(2^{k+1})$$

But $\log_2(2^k)=k$ and $\log_2(2^{k+1})=k+1$ Hence

$$k\leq\log_2x<k+1$$

By definition of the floor function, then,

$$\lfloor\log_2x\rfloor=k$$

The previouus property can be rewritten as

\vskip 1mm
If $x$ is a positive number that lies between two consecutive integer powers of $2$, the floor of the logarithm with base $2$ of $x$ is the exponent of the smaller power of $2$.

\filbreak
\vskip 1cm
{\bf Example 2: When $\lfloor\log_2(n-1)\rfloor=\lfloor\log_2n\rfloor$}

\vskip 3mm
Prove the following property:

\vskip 1mm
For any odd integer $n>1$, $\lfloor\log_2(n-1)\rfloor=\lfloor\log_2n\rfloor$

\vskip 3mm
{\bf Solution}
\vskip 1mm
If $n$ is an odd integer that is greater than $1$, then $n$ lies strictly between two successive powers of $2$:

$$2^k<n<2^{k+1}\qquad\hbox{for sime integer $k>0$}$$

It follows that $2^k\leq n-1$ because $2^k<n$ and both $2^k$ and $n$ are integers. Consequently,

$$2^k\leq n-1< 2^{k+a}$$

Applying the property from example 1 gives

$$\lfloor\log_2n\rfloor=k\qquad\hbox{and also}\quad\lfloor\log_2(n-1)\rfloor=k$$

Hence $\lfloor\log_2n\rfloor=\lfloor\log_2(n-1)\rfloor$

\filbreak
\vskip 1cm
{\bf Application: Number of Bits Needed to Represent an Integer in Binary Notation}

\vskip 3mm
Given a positive integer $n$, how many binary digits are needed to represent $n$? To answer recall that any positive integer $n$ can be written in a unique way as

$$n=2^k+c_{k-1}\cdot 2^{k-1}+\cdots c_2\cdot 2^2+c_1\cdot 2+c_0$$

where $k$ is a nonnegative integer and each $c_0,c_1,c_2\ldots c_{k-1}$ is either $0$ or $1$. Then the binary representation of $n$ is

$$1c_{k-1}c_{k-2}\cdots c_2c_1c_0$$

and so the number of binary digits needed to represent $n$ is $k+1$

\vskip 1mm
What is $k+1$ as a function of $n$? Observe that since each $c_1\leq 1$,

$$n=2^k+c_{k-1}\cdot2^{k-1}+\cdots+c_2\cdot 2^2+c_1\cdot 2+c_0\leq 2^k+2^{k-1}+\cdots 2^2+2+1$$

But by the formula for the sum of a geometric sequence

$$2^k+2^{k-1}+\cdots+2^2+2+1={2^{k+1}-1\over 2-1}=2^{k+1}-1$$

Hence, by transitivity of order

$$n\leq2^{k+1}-1<2^{k+1}$$

In addition, because each $c_i\geq 0$

$$2^k\leq 2^k+c_{k-1}\cdot 2^{k-1}+\cdots+c_2\cdot 2^2+c_1\cdot 2+c_0=n$$

Putting inequalities and together gives the double inequality

$$2^k\leq n< 2^{k+1}$$.

But then, by the property in example 1

$$k=\lfloor\log_2n\rfloor$$

Thus the number of binary bits needed to represent $n$ is $\lfloor\log_2n\rfloor+1$

\filbreak
\vskip 1cm
{\bf Example 4: A Recurrence Relation with a Logarithmic Solution}

\vskip 3mm
Define a sequence $a_1,a_2,a_3\ldots$ recursively as follows:

$$\eqalign{a_1&=1\cr
		a_k&=2a_{\lfloor k/2\rfloor}\qquad\hbox{for all integers $k\geq 2$}\cr}$$

a. Use iteration to guess an explicit formula for this sequence.

\vskip 1mm
b. Use strong mathematical induction to confirm the correctness of the formula obtained in part (a).

\vskip 3mm
{\bf Solution}

a. Begin by iterating to find the values of the first few terms of the sequence.

$$\eqalign{a_1&=1\cr
	a_2&=2a_{\lfloor 2/2\rfloor}=2a_1=2\cdot 1=2\cr
	a_3&=2a_{\lfloor 3/2\rfloor}=2a_1=2\cdot 1=2\cr
	a_4&=2a_{\lfloor 4/2\rfloor}=2a_2=2\cdot 2=4\cr
	a_5&=2a_{\lfloor 5/2\rfloor}=2a_2=2\cdot 2=4\cr
	a_6&=2a_{\lfloor 6/2\rfloor}=2a_3=2\cdot 2=4\cr
	a_7&=2a_{\lfloor 7/2\rfloor}=2a_3=2\cdot 2=4\cr
	a_8&=2a_{\lfloor 8/2\rfloor}=2a_4=2\cdot 4=8\cr
	a_9&=2a_{\lfloor 9/2\rfloor}=2a_4=2\cdot 4=8\cr
		\vdots\cr
	a_{15}&=2a_{\lfloor 15/2\rfloor}=2a_7=2\cdot 4=8\cr
	a_{16}&=2a_{\lfloor 16/2\rfloor}=2a_8=2\cdot 8=16\cr
		\vdots\cr}$$

Note that in each case when the subscript $n$ between two powers of $2$, $a_n$ equals the smaller power of $2$. More precisely:

$$\hbox{If $2^i\leq n<2^{i+1}$, then $a_n=2^i$.}$$

But since $n$ satisfies the inequality

$$2^i\leq n < 2^{i+1}$$

then by the property in example 1

$$i=\lfloor\log_2 n\rbrack$$

Substituting into If $2^i\leq n<2^{i+1}$, then $a_n=2^i$ gives

$$a_n=2^{\lfloor\log_2 n\rfloor}$$

b. The following proof shows that if $a_1,a_2,a_3,\ldots$ is a sequence of numbers that satisfies

$$a_1=1\quad a_k=2a_{\lfloor k/2\rfloor}\qquad\hbox{for all integers $k\geq 2$}$$

then the sequence satisfies the formula

$$a_n=2^{\lfloor\log_2 n\rfloor}\qquad\hbox{for all integers $n\geq 1$}$$

\vskip 3mm
{\bf Proof}
\vskip 1mm
Let $a_1,a_2,a_3,\ldots$ be the sequence defined by specifying that $a_1=1$ and $a_k=2_{\lfloor a_{k/2}\rfloor}$ for all integers $k\geq 2$, and let the property $P(n)$ be the equation

$$a_n=2^{\lfloor\log_2 n\rfloor}\qquad\quad\quad\gets P(n)$$

We will use strong mathematical induction to prove that for all integers $n\geq 1$, $P(n)$ s true

\vskip 1mm
{\bf Show that $P(1)$ is true:} By definition of $a_1,a_2,a_3,\ldots$, we have that $a_1=1$. But it is also the case that $2^{\lfloor\log_2 1\rfloor}=2^0=1$. Thus $a_1=1=2^{\lfloor\log_2 1\rfloor}$ and $P(1)$ is true.


\vskip 1mm
{\bf Show that for all integers $k\geq 1$, if $P(i)$ is true for all integers $i$ from $1$ through $k$, then $P(k+1)$ is also true:} Let $k$ be any integers with $k\geq 1$, and suppose that

$$a_i=2^{\lfloor\log_2 i\rfloor}\qquad\hbox{for all integers $i$ with $1\geq i\geq k$\qquad\quad$\gets$ inductive hypothesis}$$

We must show that

$$a_{k+1}=2^{\lfloor\log_2(k+1)\rfloor}\qquad\quad\quad\gets P(k+1)$$

\vskip 1mm
Consider the two cases: $k$ is even and $k$ is odd

\vskip 1mm
{\bf Case 1:}

$$\eqalign{a_{k+1}&=2a_{\lfloor(k+1)/2\rfloor}\cr
		&=2a_{\lfloor k/2\rfloor}\cr
		&=2\cdot 2^{\lfloor k/2\rfloor}\cr
		&=2^{\lfloor (k/2)+1\rfloor}\cr
		{}\cr
		&=2^{\lfloor\log_2 (k/2)\rfloor+1}\cr
		&=2^{\lfloor\log_2 k-\log_2 2\rfloor+1}\cr
		{}\cr
		&=2^{\lfloor\log_2 k-1\rfloor+1}\cr
		&=2^{\lfloor\log_2 k\rfloor-1+1}\cr
		&=2^{\lfloor\log_2 k\rfloor}\cr
		&=2^{\lfloor\log_2 (k+1)\rfloor+1}\cr}$$

{\bf Case 2 ($k$ is odd)}

\vskip 2mm
Thus in either case, $a_n=2^{\lfloor\log_2(k+1)\rfloor}$

\filbreak
\vskip 1cm
{\bf Exponential and Logarithmic Orders}
\vskip 3mm
How do graphs of logarithmic and exponential functions compare with graphs of power functions? It turns out that for large values of $x$, the graph of the logarithmic function with any base $b>1$ lies below the graph of any positive power function, and the graph of the exponential function with any base $b>1$ lies above the graph of any positive power function. In analytic terms, this says the following:

\vskip 2mm
For all real numbers $b$ and $r$ with $b>1$ and $r>0$

$$\log_b x\leq x^r\qquad\hbox{for all sufficiently large real numbers $x$}$$

and

$$x^r\leq b^x\qquad\hbox{for all sufficiently large real numbers of $x$}$$

These statements have the following implications for $O$-notation.

\vskip 3mm
For all real numbers $b$ and $r$ with $b>1$ and $r>0$

$$\log_bx\qquad\hbox{is}\quad O(x^r)$$

and

$$x^r\qquad\hbox{is}O(b^x)$$

\vskip 2mm
Another important function in the analysis of algorithms is the function $f$ defined by the formula

$$f(x)=x\log_bx\qquad\hbox{for all real numbers $x>0$}$$

For large values of $x$, the graph of this function fit in between the graph of the identity function and the graph of the squaring function. More precisely:

\vskip 2mm
For all real numbers $b$ with $b>1$ and for all sufficiently large real numbers $x$,

$$x\leq x\log_bx\leq x^2$$.

\vskip 2mm

The $O$-notation versions of these facts are as follows:

\vskip 2mm
For all real numbers $b>1$,

$$x\quad\hbox{is}\quad O(x\log_bx)\quad\hbox{and}\quad x\log_bx\quad\hbox{is}\quad O(x^2)$$

\filbreak
\vskip 1cm
{\bf Example 5: Deriving an Order from Logarithmic Inequalities}

\vskip 3mm
Show that $x+x\log_2x$ is $\Theta(x\log_2x)$.

\vskip 3mm
First observe that $x+x\log_2x$ is $\Omega(x\log_2x)$ because for all real numbers $x>1$,

$$x\log_2x\leq x+x\log_2x$$

and since all quantities are positive,

$$|x\log_2x|\leq|x+x\log_2x|$$

Let $A=1$ and $a=1$. Then

$$A|x\log_2x|\leq|x+x\log_2x|\qquad\hbox{for all $x>a$}$$

Hence, by definition of $\Omega$-notation

$$x+x\log_2x\quad\hbox{is}\quad\Omega(x\log_2x)$$

To show that $x+x\log_2x$ is $O(x\log_2x)$, note that according to the property for all real numbers $b$ with $b>1$ with $b=2$, there is a number $b$ such that for all $x>b$

$$\eqalign{x&<x\log_2x\cr
	\Rightarrow x+x\log_2x&< 2x\log_2x}$$

Thus, if $b$ is taken to be greater than $2$, then

$$|x+x\log_2x|<2|x\log_2x|$$

Let $B=2$. Then

$$|x+x\log_2x|\leq B|x\log_2x|\qquad\hbox{for all $x>b$}$$

Hence, by definition of $O$-notation

$$x+x\log_2x\quad\hbox{x}\quad O(x\log_2x)$$

Therefore, since $x+x\log_2x$ is $\Omega(x\log_2x)$ and $x+x\log_2x$ is $O(x\log_2x)$, by Theorem: Properties of $O$-,$\Omega$-, and $\Theta$- notations

$$x+x\log_2x\quad\hbox{is}\quad \Theta(x\log_2x)$$

\filbreak
\vskip 1cm
{\bf Example 6: Logarithm Base $b$ Is Big-Theta of Logarithm with Base $c$}

\vskip 3mm
Show that if $b$ and $c$ are real numbers such that $b>1$ and $c>1$, then $\log_bx$ is $\Theta(\log_cx)$.

\vskip 3mm
{\bf Solution}
\vskip 1mm
Suppose $b$ and $c$ are real numbers and $b>1$ and $c>1$. To show that $\log_bx$ is $\Theta(\log_cx)$, positive real numbers $A,B$, and $k$ must be found such that

$$A|\log_cx|\leq|\log_bx|\leq B|\log_cx|\qquad\hbox{for all real numbers $x>k$.}$$

by part (d) of the Properties of Logarithms

$$\log_bx={\log_cx\over\log_cb}=\Biggl({1\over\log_c b}\Biggr)\log_cx$$

Since $b>1$ and the logarithmic function with base $c$ is strictly increasing, then $\log_cb>\log_c1=0$, and so ${1\over\log_cb}>0$ also. Futhermore, if $x>1$, then $\log_bx>0$ and $\log_cx>0$.
\vskip 1mm

It follows from the equation for $\log_bx$, therefore that

$$\Biggl({1\over\log_cb}\Biggr)\log_cx\leq\log_bx\leq\Biggl({1\over\log_cb}\Biggr)\log_cx$$

for all real numbers $x>1$. Accordingly, let $A={1\over\log_cb}, B={1\over\log_cb}$, and $k=1$. Then, since all quantities in the above inequality are positive,

$$A|\log_cx|\leq|\log_bx|\leq B|\log_cx|\qquad\hbox{for all real numbers $x>k$}$$

Hence, by definition of $\Theta$-notation,

$$\log_bx\qquad\hbox{is}\quad\Theta(\log_cx)$$

\filbreak
\vskip 1cm
Example 7 shows how a logarithmic order can arise from the computation of a certain kind of sum. It requires thw following fact from calculus:

The area underneath the graph of $y=1/x$ betwen $x=1$ and $x=n$ equals $\ln n$, where $\ln n=\log_cn$.

\vskip 1cm
{\bf Example 7: Order of a Harmonic Sum}

\vskip 3mm
Sums of the form $1+{1\over 2}+\cdots+{1\over n}$ are called harmonic sums. They ocur in the analysis of various computer algorithms such as quick sort. Show that $1+{1\over 2}+{1\over 3}\cdots+{1\over n}$ is $\Omega(\ln n)$ by performing the steps:

\vskip 1mm
Show that 

\vskip 1mm

a. $${1\over 2}+{1\over 3}\cdots+{1\over n}\leq\ln n$$

and

$$\ln n\leq 1+{1\over 2}+{1\over 3}\cdots+{1\over n}$$
b. Show that if $n$ is an integer that is at least $3$, then $1\leq \ln n$.

\vskip 3mm
c. Deduce from (a) and (b) that if the inteer $n$ is greater than or equal to $3$, then

$$\ln n\leq 1+{1\over 2}+{1\over 3}+\cdots+{1\over n}\leq2\ln n$$

\vskip 3mm
d. Deduce from (c) that

$$1+{1\over 2}+{1\over 3}+\cdots+{1\over n}\quad\hbox{is}\quad\Theta(\ln n)$$

\vskip 3mm
{\bf Solution}

\vskip 1mm
To find the area of a curve we can take the left and right riemann sum. The rectangle whose bases are the intervals between each pair of integers from $1$ to $n$ and whose heights of the graph of $y=1/x$ above the right-hand endpoints of the intervals is the Right Hand Riemann Summ. Rectangles with the same bases but whose heights are the heights of the graph above the left-hand endpoints of the intervals is Left Hand Riemann Summ.

\vskip 1mm
Now the area of each rectangle is its base times its height. Since all the rectangles have base $1$, the area of each rectangle equals its height. Thus

$$\hbox{the area of the rectangle from $1$ to $2$ is }{1\over 2}$$

$$\hbox{the area of the rectangle from $2$ to $3$ is }{1\over 3}$$
\centerline{$\vdots$}
$$\hbox{the area of the rectangle from $n-1$ to $n$ is }{1\over n}$$

So the sum of the areas of all rectangles is ${1\over 2}+{1\over 3}+\cdots+{1\over n}$. This sum is less than the area underneath the graph of $f$ between $x=1$ and $x=n$, which is known to equal $n$. Hence

$${1\over 2}+{1\over 3}+\cdots+{1\over n}\leq n$$

A similiar analysis of the areas of the combined Left and Right Summ shows that

$$\ln n\leq 1+{1\over 2}+{1\over 3}+\cdots+{1\over n}$$

b. Suppose $n$ is an integer and $n\geq 3$. Since $e\approx 2.718$, then $n\geq e$. Now the logarithmic function with base $e$ is strictly increasing. Thus since $e\leq n$, then $1=\ln e\leq\ln n$

\vskip 2mm
c. By part (a)

$${1\over 2}+{1\over 3}+\cdots+{1\over n}\leq\ln n$$

and by part (b)

$$1\leq\ln n$$

Adding these two inequalities together gives

$$1+{1\over 2}+{1\over 3}+\cdots+{1\over n}\leq 2\ln n\qquad\hbox{for any integer $n\geq 3$}$$

d. Putting together the results of parts (a) and (c) leads to the conclusion that for all integers $n\geq 3$,

$$\ln n\leq 1+{1\over 2}+{1\over 3}+\cdots+{1\over n}\leq 2\ln n$$

And because all the quantities are positive for $n\geq 3$,

$$|\ln n|\leq \Biggl|1+{1\over 2}+{1\over 3}+\cdots+{1\over n}\Biggr|\leq 2|\ln n|$$

Let $A=1$, $B=2$, and $k=3$. Then

$$A|\ln n|\leq\Biggl|1+{1\over 2}+{1\over 3}+\cdots+{1\over n}\leq 2\ln n\Biggr|\leq B|\ln n|\qquad\hbox{for all $n>k$.}$$

Hence by definition of $\Theta$-notation,

$$1+{1\over 2}+{1\over 3}+\cdots+{1\over n}\quad\hbox{is}\quad\Theta(\ln n)$$


\vfill\eject
\bye
