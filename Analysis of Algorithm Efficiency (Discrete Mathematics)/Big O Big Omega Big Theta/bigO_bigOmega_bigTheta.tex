{\bf $O$-, $\Omega$-, and $\Theta$- Notations}
\vskip 1mm
\hrule

\vskip 1cm
Suppose $f$ and $g$ are real-valued functions of a variable $x$.

\vskip 3mm
1. If, for sufficiently large values of $x$, the values of $|f|$ are less than those of a multiple of $|g|$, then $f$ is of ordee at most $g$, or $f(x)$ is $O(g(x))$.

\vskip 3mm
2. If, for sufficiently large values of $x$, the values of $|f|$ are greater than those of a multiple of $|g|$, then $f$ is of order at least $g$, or $f(x)$ is $\Omega(g(x))$.

\vskip 3mm
3. If, for sufficiently large values of $x$, the values of $|f|$ are bounded above and below by those of multiples of $|g|$, then $f$ is of order $g$, or $f(x)$ is $\Theta(g(x))$.

\vskip 3mm
{\bf Definition}

\vskip 1mm
Let $f$ and $g$ be real-valued functions defined on the same set of nonnegative real numbers. Then

\vskip 3mm
1. {\bf $f$ is of order at least $g$}, written {\bf $f(x)$ is $\Omega(g(x))$}, if and only if, if there exist a positive real number $A$ and a nonnegative real number $a$ such that

$$A|g(x)|\leq |f(x)|\qquad\hbox{for all real numbers $x>a$}$$

\vskip 3mm
2. {\bf $f$ is of order at most $g$}, written {\bf $f(x)$ is $O(g(x))$}, if and only if, there exists a positive real number $B$ and a nonnegative real number $b$ such that

$$|f(x)|\leq B|g(x)|\qquad\hbox{for all real numbers $x>b$}$$

\vskip 3mm
3. {\bf $f$ is of order $g$}, written {\bf $f(x)$ is $\Theta(g(x))$}, if and only if, there exists a positive real number $A,B$, and a non-negative real number $k$ such that

$$A|g(x)|\leq|f(x)|\leq B|g(x)|\qquad\hbox{for all real numbers $x>k$}$$

\filbreak
\vskip 1cm
{\bf Example 1: Translating to $\Theta$-Notation}

\vskip 1mm
Use $\Theta$-Notation

$$10|x^6|\leq |17x^6-45x^3+2x+8|\leq 30|x^6|\qquad\hbox{for all real numbers $x>2$}$$

\vskip 3mm
{\bf Solution}

\vskip 1mm
Let $A=10,B=30$, and $k=2$. Then the statement translates to

$$A|x^6|\leq|17x^6-45x^3+2x|\leq B|x^6|\qquad\hbox{for all real numbers $x>k$}$$

so by definition of $\Theta$-notation

$$17x^6-45x^3+2x+8\hbox{ is }\Theta(x^6)$$

\filbreak
\vskip 1cm
{\bf Example 2: Translating to $O$- and $\Omega$- Notations}

\vskip 1mm
a. Use $\Omega$ and $O$ notations to express the statements

$$\hbox{\rm (i) }\quad 15\bigl|\sqrt{x}\bigr|\leq\Biggl|{14\sqrt{x}(2x+9)\over x+1}\Biggr|\qquad\hbox{for all real numbers $x>0$}$$

$$\hbox{\rm (ii) }\quad \Biggl|{15\sqrt{x}(2x+9)\over x+1}\Biggr|\leq 45\bigl|\sqrt{x}\bigr|\qquad\hbox{for all real numbers $x>7$}$$

\vskip 2mm
b. Justify the statement:

$${15\sqrt{x}(2x+9)\over x+1}\hbox{ is } \Theta(\sqrt{x})$$

\vskip 3mm
{\bf Solution}

\vskip 1mm
a. {\rm (i)} Let $A=15$ and $a=0$. The given statement translates to

$$A|\sqrt{x}\leq \Biggl|{15\sqrt{x}(2x+9)\over x+1}\Biggr|\qquad\hbox{for all real numbers $x>a$}$$

so by definition of $\Omega$-notation

$${15\sqrt{x}(2x+9)\over x+1}\hbox{ is }\Omega(\sqrt{x})$$

\vskip 3mm
{\rm (ii)} Let $B=45$ and $b=7$. The given statement translates to

$$\Biggl|{15\sqrt{x}(2x+9)\over x+1}\Biggr|\hbox{ is } O(\sqrt{x})$$

\vskip 3mm
b. Let $A=15, B=45$ and let $k$ be the larger of $0$ and $7$. Then when $x>k$, both inequalities in {\rm (i)} and {\rm (ii)} are satisfied, and so

$$A|\sqrt{x}|\leq \Biggl|{15\sqrt{x}(2x+9)\over x+1}\Biggr| \leq B|\sqrt{x}|\qquad\hbox{for all real numbers $x>k$}$$

hence by definition of $\Theta$-notation

$${15\sqrt{x}(2x+9)\over x+1}\hbox{ is } \Theta(\sqrt{x}$$

\filbreak
\vskip 1cm
{\bf Orders of Power Functions}

\vskip 1mm
Observe that if

$$1<x$$

then

$$x<x^2\qquad\hbox{\bf multiply both sides by $x$}$$

and so

$$x^2<x^3\qquad\hbox{\bf multiply both sides by $x$}$$

For any rational number $r$ and $s$

$$\hbox{if $x>1$ and $r<s$, then $x^r<x^s$}$$

This property has the following consequence for orders

For any rational numbers $r$ and $s$,

$$\hbox{if $r<s$, then $x^r$ is $O(x^s)$}$$

\filbreak
\vskip 1cm
{\bf Orders of Polynomial Functions}

\vskip 3mm
{\bf Example 3: A Polynomial Inequality}

\vskip 1mm
Show that for any real number $x$,

$$\hbox{if} x>1\quad\hbox{then}\quad 3x^3+2x+7\leq 12x^3$$

\vskip 3mm
{\bf Solution}

\vskip 1mm
Suppose $x$ is a real number and $x>1$. Then by the orders of powers property ,

$$x<x^3\quad\hbox{and}\quad 1<x^3$$

Multiply the left-hand inquality by $2$ and the right-hand inequality by $7$ to get

$$2x<2x^3\quad\hbox{and}\quad 7<7x^3$$

Now add $3x^3\leq3x^3, 2x<2x^3$, and $7<7x^3$ to obtain

$$3x^3+2x+7\leq 3x^3+2x^3+7x^3=12x^3$$

\filbreak
\vskip 1cm
{\bf Example 4: Using the Definitions to Show That a Polynomial Function with Positive Coefficients Has a Cetain Order}

\vskip 1mm
Use the definitions of big-Omega, big-O, and big-Theta to show that $2x^4+3x^3+5$ is $\Theta(x^4)$

\vskip 3mm
{\bf Solutions}

\vskip 1mm
Define functions $f$ and $g$ as follows. For all real numbers $x$

$$\eqalign{f(x)&=2x^4+3x^3+5\cr
	g(x)=x^4\cr}$$

Observe that for all real numbers $x>0$

$$2x^4\leq 2x^4+3x^3+5\qquad\hbox{\bf because $3x^3+5>0$ for $x>0$}$$

and so

$$2|x^4|\leq|2x^4+3x^3+5|\qquad\hbox{\bf because all terms on both sides of the inequality are positive}$$

Let $A=2$ and $a=0$. Then

$$A|x^4|\leq|2x^4+3x^3+5\qquad\hbox{for all $x>a$}$$

and so by definition of $\Omega$-notation, $2x^4+3x^3+5$ is $\Omega(x^4)$.

\vskip 1mm
Also for $x>1$

$$\eqalign{2x^4+3x^3+h&\leq 2x^4+3x^4+5x^4\cr
	2x^4+2x^3+5&\leq 10x^4\cr
	|2x^4+3x^3+5|&\leq 10|x^4|\cr}$$

Let $B=10$ and $b=1$. Then

$$|2x^4+3x^3+5|\leq B|x^4|\qquad\hbox{for all $x>b$}$$

and so, by definition of $O$-notation, $2x^4+3x^3+5$ is $O(x^4)$.

\vskip 1mm
Since $2x^4+3x^3+5$ is both $\Omega(x^4)$ and $O(x^4)$ By the Properties of $O$-, $\Omega$-, and $\Theta$-Notations

\filbreak
\vskip 1cm
{\bf Example 5: A Big-$O$ Approximation for a polynomial with Some Negative Coefficients}

\vskip 3mm
a. Use the definition of $O$-notation to show that $3x^3-1000x-200$ is $O(x^3)$

\vskip 2mm
b. Show that $3x^3-1000x-200$ is $O(x^s)$ for all integers $s>3$.

\vskip 3mm
{\bf Solution}

\vskip 1mm
a. According to the triangle inequality for absolute value

$$|a+b|\leq|a|+|b|\qquad\hbox{for all real numbers $a$ and $b$}$$

If $-b$ is substituted in place of $b$, the result is

$$\eqalign{|a-b|&=|a+(-b)|\leq|a|+|-b|=|a|+|b|\cr
		|a-b|\leq|a|+|b|}$$

It follows that for all real numbers $x>1$

$$\eqalign{|3x^3-1000x-200|&\leq |3x^3|+|1000x|+|200|\cr
		|3x^3-1000x-200|&\leq 3x^3+1000x+200\cr
		|3x^3-1000x-200|&\leq 1203x^3\cr
		|3x^3-1000x-200|&\leq 1203|x^3|}$$

Let $b=1$ and $B=1203$. Then

$$|3x^3-1000x-200|\leq B|x^3|\qquad\hbox{for all real numbers $x>b$}$$

So, by definition of $O$-notation, $3x^3-1000x-200$ is $O(x^3)$.

\vskip 3mm
b. Suppose $s$ is an integer with $s>3$. By the order of polynomial property, $x^3<x^s$ for all real numbers $x>1$. So $B|x^3|<B|x^s|$ for all real numbers $x>b$ (because $b=1$), and thus by part (a)

$$|3x^3-1000x-200|\leq B|x^s|\qquad\hbox{for all real numbers $x>b$}$$

Hence, by definition of $O$-notation, $3x^3-1000x-200$ is $O(x^s)$ for all integers $s>3$.

\filbreak
\vskip 1cm
{\bf Example 6: A Big-Omega Approximation for a Polynomial with Some Negative Coefficients}

\vskip 1mm
a. Use the definition of $\Omega$-notation to show that $3x^3-1000x-200$ is $\Omega(x^3)$.

\vskip 2mm
b. Show that $3x^3-1000x-200$ is $\Omega(x^r)$ for all integers $r<3$.

\vskip 3mm
{\bf Solution}

\vskip 1mm
a. To show that $3x^3-1000x-200$ is $\Omega(x^3)$, you need to find numbers $a$ and $A$ so that $A|x^3|\leq|3x^3-1000x-200|$ for all real numbers $x>a$

\vskip 1mm
Choose $a$ as follows: Add up the absolute values of the coeeficiencts of the lower order terms of $3x^3-1000x-200$, divide by the absolute value of the highest-power term, and multiply the result by $2$. The result is $a=2(1000+200)/3$, which equals $800$. $A$ can be taken to be one-half of the absolute value of the highest power of the polynomial.

$$\eqalign{x&> 800\cr
	x&>2\Biggl({1000+200\over 3}\Biggr)\cr
	x&>\Biggl({2\cdot 1000\over 3}\Biggr)+\Biggl({2\cdot 200\over 3}\Biggr)\cr
	x&>\Biggl({2\cdot 1000\over 3}\cdot{1\over x}\Biggr)+\Biggl({2\cdot 200\over 3}\cdot{1\over x^2}\Biggr)\cr
	{3\over 2}x^3&>1000x+200\cr
	3x^3-{3\over 2}x^3&>1000x+200\cr
	3x^3-1000x-200&>{3\over 2}x^3\cr
	|3x^3-1000x-200|&>{3\over 2}|x^3|\cr}$$

Let $A={3\over 2}$ and let $a=800$. Then

$$A|x^3|\leq |3x^3-1000x-200|\qquad\hbox{for all real numbers $x>a$}$$

So, by definition of $\Omega$-notations, $3x^3-100x-200$ is $\Omega(x^3)$

\vskip 3mm
b. Suppose $r$ is an integer with $r<3$. By the order ofpolynomials property , $x^r<x^3$ for all real numbers $x>1$. So, since $a=800>1$, $A|x^r|<A|x^3|$ for all real numbers $x>a$. Thus, by part (a)

$$A|x^r|\leq|3x^3-1000x-200|\qquad\hbox{for all real numbers $x>a$}$$

Hence, by definition of $\Omega$-notations, $3x^3-1000x-200$ is $\Omega(x^r)$ for all integers $r<3$.

\filbreak
\vskip 1cm
{\bf Example 7: Calculating Polynomial Orders Using the Theorem on Polynomial Orders}

\vskip 1mm
Use the theorem on polynomial orders to find orders for the functions given by the following formulas.

\vskip 3mm
a.

$$f(x)=7x^5+5x^3-x+4\qquad\hbox{for all real numbers $x$}$$

b.

$$g(x)={(x-1)(x+1)\over 4}\qquad\hbox{for all real numbers $x$}$$

\vskip 3mm
{\bf Solution}

\vskip 1mm
a. By direct application of the theorem on polynomial orders, $7x^5+5x^3-x+4$ is $\Theta(x^5)$

\vskip 3mm
b. $$\eqalign{g(x)&={(x-1)(x+1)\over 4}\cr
		&={1\over 4}(x^2-1)\cr
		&={1\over 4}x^2-{1\over 4}\cr}$$

Thus $g(x)$ is $\Theta(x^2)$ by the theorem on polynomial orders.

\filbreak
\vskip 1cm
{\bf Example 8: Showing That Two Power Functions Have Different Orders}

\vskip 1mm
Show that $x^2$ is not $O(x)$, and deduce that $x^2$ is not $\Theta(x)$.

\vskip 3mm
{\bf Solution [argue by contradiction]}

\vskip 1mm
Suppose that $x^2$ is $O(x)$. By the supposition that $x^2$ is $O(x)$, there exist a positive real number $B$ and a nonnegative real number $b$ such that

$$|x|\leq B|x|\qquad\hbox{for all real numbers $x>b$}\qquad\hbox{(1)}$$

Let $x$ be a positive real number that is greater than both $B$ and $b$. Then

$$\eqalign{x\cdot x&>B\cdot x\cr
	|x^2|&>B|x|\cr}$$

Thus there is a real number $x>b$ such that

$$|x^2|>B|x|$$

This contradicts (1). Hence the supposition is false, and $x^2$ is not $O(x)$. By the Properties of $O$-,$\Omega$-, and $\Theta$- Notations theorem, if $x^2$ is $\Theta(x)$, then $x^2$ is $O(x)$. But $x^2$ is not $O(x)$, and thus $x^2$ is not $\Theta(x)$.

\filbreak
\vskip 1cm
{\bf Orders for Functions of Integer Variables}

\vskip 1mm
It is traditional to use the symbol $x$ to denote a real number variable, whereas $n$ is used to represent an integer variable. Thus, given a statement of the form

$$f(n)\hbox{ is } \Theta(g(n))$$

we assume that $f$ and $g$ are functions defined on sets of integers. If it is true that

$$f(x)\hbox{ is } \Theta(g(x))$$

where $f$ and $g$ are functions defined for real numbers, then it is certainly true that $f(n)$ is $\Theta(g(n))$. The reason is that if $f(x)$ is $\Theta(g(x))$, then an inequality

$$A|g(x)|\leq|f(x)|\leq B|g(x)|$$

holds for all real numbers $x>k$. Hence, in particular, the inequality

$$A|g(n)|\leq|f(n)|\leq B|g(n)|$$

holds for all integers $n>k$

\filbreak
\vskip 1cm
{\bf Example 9: An Order for the Sum of the First $n$ Integers}

\vskip 1mm
Sums of the form $1+2+3+\cdots+n$ arise in the analysis of computer algorithms such as selection sort. Show that for a positive integer variable $n$,

$$1+2+3+\cdots+n\hbox{ is } \Theta(n^2)$$

\vskip 3mm
{\bf Solution}

\vskip 1mm
By the formula for the sum of the first $n$ integers, for all positive integers $n$,

$$1+2+3+\cdots+n={n(n+1)\over 2}$$

But

$${n(n+1)\over 2}={1\over 2}n^2+{1\over 2}n$$

And, by the theorem on polynomial orders,

$${1\over 2}n^2+{1\over 2}n\hbox{ is } \Theta(n^2)$$

Hence

$$1+2+3+\cdots+n\hbox{ is } \Theta(n^2)$$

\filbreak
\vskip 1cm
{\bf Extension to Functions Composed of Rational Power Functions}

\vskip 1mm
Consider a function of the form

$${(x^{3/2}+3)(x-2)^2\over x^{1/2}(2x^{1/2}+1)}={x^{7/2}-4x^{5/2}+4x^{3/2}+3x^2-12x+12\over 2x+x^{1/2}}$$

When the numerator and denominator are expanded, each is a sum of terms of the form $ax^r$, where $a$ is a real number and $r$ is a positive rational number. The degree of such a sum can be taken to be the largest exponent of $x$ that occurs in one of its terms. If the difference between the degree of the numerator and that of the denominator is called the degree of the function and denoted $d$, then it can be shown that $f(x)$ is $\Theta(x^d)$, that $f(x)$ is $O(x^c)$ for all real numbers $c>d$, and that $f(x)$ is not $O(x^c)$ for any real number $c<d$. This means that for ths example $d=7/2-1=5/2$ and that

$${(x^{3/2}+3)(x-2)^2\over x^{1/2}(2x^{1/2}+1)}\hbox{ is } \Theta(x^{5/2})$$

$${(x^{3/2}+3)(x-2)^2\over x^{1/2}(2x^{1/2}+1)}\hbox{ is } O(x^c)\qquad\hbox{for all real numbers $c>5/2$}$$

and

$${(x^{3/2}+3)(x-2)^2\over x^{1/2}(2x^{1/2}+1)}\hbox{ is not } O(x^c)\qquad\hbox{for all real numbers $c>5/2$}$$


\filbreak
\vfill\eject
\bye
