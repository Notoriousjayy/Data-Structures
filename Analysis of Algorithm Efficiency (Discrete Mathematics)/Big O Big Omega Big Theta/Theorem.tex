{\bf Theorem: Properties of $O$-, $\Omega$-, and $\Theta$- Notations}
\vskip 1mm
\hrule

\vskip 3mm
Let $f$ and $g$ be real-valued functions defined on the same set of nonnegative real numbers.

\vskip 1mm
1. $f(x)$ is $\Omega(g(x))$ and $f(x)$ is $O(g(x))$ if, and only if $f(x)$ is $\Theta(g(x))$.

\vskip 3mm
2. $f(x)$ is $\Omega(g(x))$ if, and only if, $g(x)$ is $O(f(x))$.

\vskip 3mm
3. If $f(x)$ is $O(g(x))$ and $g(x)$ is $O(h(x))$, then $f(x)$ is $O(h(x))$.

\vskip 3mm
{\bf Proof}

\vskip 1mm
1. We first show that if $f(x)$ is $\Omega(g(x))$, then $g(x)$ is $O(f(x))$. Thus, suppose $f(x)$ is $\Omega(g(x))$. By definition of $\Omega$-notation, there exist a positive real number $A$ and a nonnegative real number $a$ such that

$$A|g(x)|\leq|f(x)|\qquad\hbox{for all real numbers $x>a$}$$

Divide both sides by $A$ to obtain

$$|g(x)|\leq{1\over A}|f(x)|\qquad\hbox{for all real numbers $x>a$}$$

Let $B=1/A$ and $b=a$. Then $B$ is a positive real number and $b$ is a nonnegative real number, and

$$|g(x)|\leq B|f(x)|\qquad\hbox{for all real numbers $x>b$}$$

and so $g(x)$ is $O(f(x))$.

\vskip 3mm
3. Suppose $f(x)$ is $O(f(x))$ and $g(x)$ is $O(h(x))$. By definition of $O$-notation, there exist positive real numbers $B_1$ and $B_2$, and nonnegative real numbers $b_1$ and $b_2$ such that

$$|f(x)|\leq B_1|g(x)|\qquad\hbox{for all real numbers $x>b_1$}$$

and

$$|g(x)|\leq B_2|h(x)|\qquad\hbox{for all real numbers $x>b_2$}$$

Let $B=B_1B_2$, and let $b$ be the greater of $b_1$ and $b_2$. Then if $x>b$,

$$|f(x)|\leq B_1|g(x)|\leq B_1(B_2|h(x)|)\leq B|h(x)|$$

Thus, by definition of $O$-notation, $f(x)$ is $O(h(x))$.

\filbreak
\vskip 1cm
{\bf Theorem: On Polynomial Orders}
\vskip 1mm
\hrule

\vskip 3mm
Suppose $a_0,a_1,a_2,\ldots,a_n$ are real numbers and $a_n\neq 0$.

\vskip 1mm
1. $a_nx^n+a_{n-1}x^{n-1}+\cdots+a_1x+a_0$ is $O(x^s)\quad$ for all integers $s\geq n$.

\vskip 3mm
2. $a_nx^n+a_{n-1}x^{n-1}+\cdots+a_1x+a_0$ is $\Omega(x^r)\quad$ for all integers $r\leq n$.

\vskip 3mm
3. $a_nx^n+a_{n-1}x^{n-1}+\cdots+a_1x+a_0$ is $\Theta(x^n)$

\filbreak
\vskip 1cm
{\bf Theorem: Limitation on Orders of Polynomial Functions}
\vskip 1mm
\hrule

\vskip 1mm
Let $n$ be a positive integer, and let $a_0,a_1,a_2,\ldots,a_n$ be real numbers with $a_n\neq 0$. If $m$ is any integer with $m<n$, then

$$a_nx^n+a_{n-1}x^{n-1}+\cdots+a_1x+a_0\hbox{ is not } O(x^m)$$

and

$$a_nx^n+a_{n-1}x^{n-1}+\cdots+a_1x+a_0\hbox{ is not } \Theta(x^m)$$

\filbreak
\vskip 1cm
{\bf Theorem: Orders of Functions Composed of Rational Power Functions}
\vskip 1mm
\hrule

\vskip 1mm
Let $m$ and $n$ be positive integers, and let $r_0,r_1,r_2,\ldots r_n$ and $s_0,s_1,s_2,\ldots,s_m$ be nonnegative rational numbers with $r_0<r_1<r_2<\cdots<r_n$ and $s_0<s_1<s_2<\cdots<s_m$. Let $a_0, a_1,a_2,\ldots,a_n$ and $b_0,b_1,b_2,\ldots,b_m$ be real numbers with $a_n\neq 0$ and $b_m\neq 0$. Then

$${a_nx^{r_n}+a_{n-1}+\cdots+a_1x^{r_1}+a_0x^{r_0}\over b_mx^{s_m}+b_{m-1}x^{s_1}+b_0x^{s_0}}\hbox{ is } \Theta(x^{r_n-s_m})$$

$${a_nx^{r_n}+a_{n-1}+\cdots+a_1x^{r_1}+a_0x^{r_0}\over b_mx^{s_m}+b_{m-1}x^{s_1}+b_0x^{s_0}}\hbox{ is } O(x^c)\qquad\hbox{for all real numbers $c>r_n-s_m$}$$

$${a_nx^{r_n}+a_{n-1}+\cdots+a_1x^{r_1}+a_0x^{r_0}\over b_mx^{s_m}+b_{m-1}x^{s_1}+b_0x^{s_0}}\hbox{ is not } O(x^c)\qquad\hbox{for any real numbers $c<r_n-s_m$}$$



\filbreak
\vfill\eject
\bye
