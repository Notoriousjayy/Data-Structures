{\bf Mathematical Induction I}
\vskip 1mm
\hrule

\vskip 1cm
The validity of proof by mathematical induction is generally taken as an axiom. That is why it is referred to as the principle of mathematical induction rather than as a theorem. It is equivalent to the following property of the integers, which is easy to accept.

\vskip 1mm
Suupose $S$ is any set of integers satisfying the statement $P(a)$ is true. Represented in set notation as $a\in S$ and for all integers $k\geq a$, if $k \in S$ then $k+1\in S$. Then $S$ must contain every integer greater than or equal to $a$.

\vskip 1mm
Proving a statement by mathematical induction is a two-step process. The first step is called the {\bf basis step}, and the second step is called the {\bf inductive step}.

\filbreak
\vskip 1cm
{\bf Theorem: Sum of the First $n$ Integers}

\vskip 1mm
For all integers $n\geq 1$,

$$1+2+\cdots+n={n(n+1)\over 2}$$

\vskip 3mm
{\bf Proof (by mathematical induction):}

\vskip 1mm
Let the property $P(n)$ be the equation

$$1+2+3+\cdots+n={n(n+1)\over 2}$$

{\bf Show that $P(1)$ is true:}

\vskip 1mm
To establish $P(1)$, we must show that

$$1={1(1+1)\over 2}$$

But the left-hand side of this equation is $1$ and the right-hand side is

$${1(1+1)\over 2}={2\over 2}= 1$$

also. Hence $P(1)$ is true.

\vskip 2mm
{\bf Show that for all integers $k\geq 1$ if $P(k)$ is true then $P(k+1)$ is also true:}

\vskip 1mm
Suppose that $k$ is any integer with $k\geq 1$ such that

$$1+2+3+\cdots+k={k(k+1)\over 2}\qquad\leftarrow\hbox{\bf $P(k)$ inductive hypothesis}$$

We must show that

$$1+2+3+\cdots+(k+1)={(k+1)[(k+1)+1]\over 2}$$

or equivalently, that

$$1+2+3+\cdots+(k+1)={(k+1)(k+2)\over 2}$$

The left-hand side of $P(k+1)$ is

$$\eqalign{1+2+3+\cdots+(k+1)&= 1+2+3+\cdots+k+(k+1)\qquad\leftarrow\hbox{\bf make the next-to-last term explicit}\cr
				&={k(k+1)\over 2}+(k+1)\qquad\leftarrow\hbox{\bf substitution from the inductive hypothesis}\cr
				&={k(k+1)\over 2}+{2(k+1)\over 2}\cr
				&={k^2+k\over 2}+{2(k+1)\over 2}\cr
				&={k^2+k\over 2}+{2k+2\over 2}\cr
				&={k^2+3k+2\over 2}\cr}$$

and the right-hand side of $P(k+1)$ is

$${(k+1)(k+2)\over 2}={k^2+3k+2\over 2}$$

Thus the two sides of $P(k+1)$ are equal to the same quantity and so they are equal to each other. Therefore the equation $P(k+1)$ is true.

\filbreak
\vskip 1cm
{\bf Theorem: Sum of a Geometric Sequence}

\vskip 1mm
For any real unmber $r$ except $1$, and any integer $n\geq 0$,

$$\sum^n_{i=0}r^i={r^{n+1}-1\over r-1}$$

\vskip 3mm
{\bf Proof (by mathematical induction):}

\vskip 1mm
Suppose $r$ is a particular but arbitrarily chosen real number that is not equal to $1$, and let the property $P(n)$ be the equation

$$\sum^n_{i=0}r^i={r^{n+1}-1\over r-1}\qquad\leftarrow\hbox{\bf $P(0)$}$$

The left-hand side of this equation is $r^0=1$ and the right-hand side is

$${r^{0+1}-1\over r-1}={r-1\over r-1}=1$$

also because $r^1=r$ and $r\neq 1$. Hence $P(0)$ is true.

\vskip 1mm
{\bf Show that for all integers $k\geq 0$, if $P(k)$ is true then $P(k+1)$ is also true:}
\vskip 1mm
Let $k$ be any integer $k\geq 0$, and suppose that

$$\sum^n_{i=0}r^i{r^{k+1}-1\over r-1}\qquad\leftarrow\hbox{\bf $P(k)$ inductive hypothesis}$$

We must show that

$$\sum^{k+1}_{i=0}r^i={r^{(k+1)+1}-1\over r-1}$$

or, equivalently, that
$$\sum^{k+1}_{i=0}r^i={r^{k+2}-1\over r-1}\qquad\leftarrow\hbox{\bf $P(k+1)$}$$

The left-hand side of $P(k+1)$ is

$$\eqalign{\sum^{k+1}_{i=0}r^i&=\sum^{k+1}_{i=0}r^i+r^{k+1}\qquad\leftarrow\hbox{\bf write the $(k+1)$st term separetly from the first $k$ terms}\cr
				&={r^{k+1}-1\over r-1}+r^{k+1}-1\qquad\leftarrow\hbox{\bf substitute from the inductive hypothesis}\cr
				&={r^{k+1}-1\over r-1}+{r^{k+1}(r-1)\over r-1}\qquad\leftarrow\hbox{\bf find common denominator}\cr
				&={(r^{k+1}-1)+r^{k+1}-(r-1)\over r-1}\qquad\leftarrow\hbox{\bf add factors}\cr				
				&={(r^{k+1}-1)+r^{k+2}-r^{k+1}\over r-1}\cr
				&={r^{k+2}-1\over r-1}\cr}$$

which is the right-hand side of $P(k+1)$.

\vfill\eject
\bye
