\nopagenumbers
{\bf Theorem: Loop Invariant Theorem}
\vskip 1mm
\hrule

\vskip 6pt
Let a {\bf while} loop with guard $G$ be given, together with pre- and post-conditions that are predicates in the algorithm variables. Also let a predicate $I(n)$, called the {\bf loop invariant}, be given. If the following four properties are true, then the loop is correct with respect to its pre- and post-conditions.

\vskip 3mm
{\bf I. \quad Basis Property}: The pre-condition for the loop implies that $I(0)$ is true before the first iteration of the loop.

\vskip 2mm
{\bf II. \quad Inductive Property}: For all integers $k\geq 0$, if the guard $G$ and the loop invariant $I(k)$ are both true before an iteration of the loop, then $I(k+1)$ is true after iteration of loop.

\vskip 2mm
{\bf III. Eventual Falsity of Guard}: After a finite number of iterations of the loop, the guard $G$ becomes false.

\vskip 2mm
{\bf IV. \quad Correctness of the Post-Condition}: If $N$ is the least number of iterations after which $G$ is false and $I(N)$ is true, then the values of the algorithm variables will be specified in the post-condition of the loop.

\vskip 10pt
{\bf Proof}

\vskip 6pt
The loop invariant theorem follows easily from the principle of mathematical induction. Assume that $I(n)$ is a predicate that satisfies properties $I-IV$ of the loop invariant theorem. Properties I and II are the basis and inductive steps needed to prove the truth of the following statement:

\vskip 2mm
\centerline{For all integers $n\geq $0, if the {\bf while} loop}

\vskip 1mm
\centerline{iterates $n$ times, then $I(n)$ is true.}

\vskip 1mm
Thus, by the principle of mathematical induction, since both I and II are true, the statement is also true.

\vskip 1mm
Property III says that the guard $G$ eventually becomes false. At that point the loop will have been iterated some number, say $N$, of times. Since $I(n)$ is true after the $n^{th}$ iteration for every $n\geq 0$, then $I(n)$ is true after the $N^{th}$ iteration. That is, after the $N^{th}$ iteration the guard is false and $I(N)$ is true. But this is the hypothesis of property IV, which is a if-then statement. Since statement IV is true (assumption) and its hypothesis is true (by the argument just given), it follows (by modus ponens) that its conclusion is also true. That is, the values of all algorithm variables after execution of the loop are as specified in the post-condition for the loop.

\vfill\eject
